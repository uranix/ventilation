\documentclass[12pt]{article}
\usepackage[utf8]{inputenc}
\usepackage[T2A]{fontenc}
\usepackage[english,russian]{babel}
\usepackage{amsmath,amssymb}
\usepackage{indentfirst}

\author{Цыбулин Иван}
\title{Сопряжение одномерных и трехмерных расчетных областей}

\newcommand{\pd}[2]{\frac{\partial #1}{\partial #2}}
\newcommand{\dpd}[2]{\dfrac{\partial #1}{\partial #2}}
\renewcommand{\arraystretch}{1.2}
\let\dividesymbol\div
\renewcommand{\div}{\operatorname{div}}
\newcommand{\grad}{\operatorname{grad}}
\newcommand{\bvec}[1]{\boldsymbol{\mathbf{#1}}}
\newcommand{\cutefrac}[2]{{}^{#1}\mkern-5mu/{\!}_#2}
\newcommand{\half}{{\cutefrac{1}{2}}}
\renewcommand{\iiint}{\int\hspace{-10pt}\int\hspace{-10pt}\int}
\newcommand{\oiint}{%
{}\subset\hspace{-6pt}\supset\hspace{-17.8pt}%
\int\hspace{-10pt}\int%
}%

\begin{document}
\maketitle

\section{Общие положения}
$$
y = \oiint_{\partial V} x dx + z
$$

Рассмотрим трехмерную область, в которой среда описывается трехмерными уравнениями Эйлера для нескольких компонент
\begin{equation}
\begin{aligned}
&\pd{\rho_i}{t} + \div \rho_i \bvec{u} = \dot m_i, \qquad i = 1, \dots, S\\
&\pd{\rho \bvec{u}}{t} + \div \left(\rho \bvec{u} \otimes \bvec{u} + p\hat I\right) = \rho \bvec{g}\\
&\pd{\rho E}{t} + \div (\rho E + p)\bvec{u} = \rho \bvec{u}\bvec{g} + \sum_i \dot m_i H_i,
\end{aligned}
\end{equation}
где 
\begin{gather*}
\rho = \sum_i \rho_i\\
E = \dfrac{u^2}{2} + \varepsilon\\
p = (\gamma - 1)\rho\varepsilon\\
\varepsilon = C_V T\\
\gamma = \frac{C_P}{C_V}\\
C_{V} = \frac{1}{\rho}\sum_i \frac{\rho_i R}{M_i (\gamma_i- 1)}\qquad
C_{P} = \frac{1}{\rho}\sum_i \frac{\gamma_i\rho_i R}{M_i (\gamma_i- 1)}
\end{gather*}

\section{Численный метод решения задачи в трехмерной области}
Численный метод решения задачи --- простейший конечно-объемный. Пусть $\vec{U}$ --- вектор консервативных 
переменных, а $\vec{\bvec{F}}$ --- векторы потоков по каждому направлению.
\begin{gather*}
\vec{U} = \begin{pmatrix}
\rho_1 & \dots & \rho_S & \rho u_x & \rho u_y & \rho u_z & \rho E
\end{pmatrix}\\
\vec{F}_x\big(\vec{U}\big) = \begin{pmatrix}
\rho_1 u_x & \dots & \rho_S u_x & \rho u_x^2 + p & \rho u_x u_y & \rho u_x u_z & (\rho E + p) u_x
\end{pmatrix}\\
\vec{F}_y\big(\vec{U}\big) = \begin{pmatrix}
\rho_1 u_y & \dots & \rho_S u_y & \rho u_x u_y & \rho u_y^2 + p & \rho u_y u_z & (\rho E + p) u_y
\end{pmatrix}\\
\vec{F}_z\big(\vec{U}\big) = \begin{pmatrix}
\rho_1 u_z & \dots & \rho_S u_z & \rho u_x u_z & \rho u_y u_z & \rho u_z^2 + p & (\rho E + p) u_z
\end{pmatrix}\\
\vec{f} = \begin{pmatrix}
\dot m_1 & \dots & \dot m_S & \rho g_x & \rho g_y & \rho g_z & \rho\bvec{u}\bvec{g} + \sum_i \dot 
m_i 
H_i
\end{pmatrix}
\end{gather*}

Запишем закон сохранения в конечном объеме V:
\begin{equation}
\pd{}{t} \iiint_V \vec{U} dV + \oiint_{\partial V} \bvec{n}\vec{\bvec{F}}\big(\vec{U}\big) dS = 
\iiint_V \vec{f} dV
\end{equation}

Поток на границе ячеек $\vec{U}_{i,j,k}$ и $\vec{U}_{i+1,j,k}$ вычисляется из решения задачи Римана:
\[
\bvec{n}\vec{\bvec{F}}_{i+\half,j,k} = 
\bvec{n}\vec{\bvec{F}}\left(\mathcal{R}_{\bvec{n}}\big(\vec{U}_{i,j,k},\vec{U}_{i+1,j,k}\big)\right)
\]
Для внутренних границ интеграл заменяется на сумму аналогов 
$\bvec{n}\vec{\bvec{F}}_{i+\half,j,k}$ по всем внутренним граням.

Для каждой внешней (лежащей на границе области) грани ячейки строится набор пар $\vec{U}^{(p)}, S^{(p)}$ --- состояние $\vec{U}^{(p)}$ в $p$-й соседней ячейке и площадь контакта с $p$-й ячейкой $S^{(p)}$.
Для внешних граней интеграл заменяется суммой
\[
\oiint_{\partial V_\text{ext}}  \bvec{n}\vec{\bvec{F}} dS 
= \sum_p S^{(p)} \bvec{n} \vec{\bvec{F}}
\left(\mathcal{R}_{\bvec{n}}\big(\vec{U}_{i,j,k},\vec{U}^{(p)}\big)\right)
\]

\section{Решение задачи Римана}
Решение задачи Римана для многокомпонентного газа в направлении $\bvec{n}$ строится с помощью решения задачи 
Римана о распаде между однокомпонентными газами с заданными показателями адиабаты.

Введем $\theta_i = \dfrac{\rho_i}{\rho}$. Тогда
\begin{gather*}
\pd{\rho \theta_i}{t} + \div \rho \theta_i \bvec{u} = 0\\
\theta_i\pd{\rho}{t} + \rho \pd{\theta_i}{t} + \theta_i \div \rho \theta_i \bvec{u} + 
\rho (\bvec{u} \nabla) \theta_i = 0\\
\pd{\theta_i}{t} + (\bvec{u} \nabla) \theta_i = 0
\end{gather*}
Величины $\theta$ переносятся со скоростью течения. Фактически, разрыв $\theta_i$ эволюционирует как контактный разрыв.

Разобьем скорость течения $\bvec{u}$ на нормальную и касательную к плоскости разрыва составляющие:
\[
\bvec u = \bvec n u_n + \bvec{u}_{\tau}
\]
Касательные скорости также образуют контактный разрыв. Задача сводится к решению одномерной задачи Римана с нормальными 
компонентами скорости:
\[
\left\{
\tilde\rho, \tilde{u}_n, \tilde p
\right\} = 
\mathcal{R}_{1D}\Big(
\left\{
\rho, u_n, p, \gamma
\right\}_L
,
\left\{
\rho, u_n, p, \gamma
\right\}_R
\Big)
\]

Далее, в зависимости от знака $\tilde{u}_n$ сносятся либо правые, либо левые значения $\bvec{u}_{\tau}, \theta_i, \gamma$. Из этих значений определяется весь вектор консервативных величин $\vec{U}$ на границе ячеек.

\section{Сопряжение одномерной и многомерной области}

Рассмотрим одномерную трубу, состоящую из одной цепочки трехмерных ячеек. Пусть на границах трубы стоят отражающие 
граничные условия. Потоки на границе получаются из решения задачи Римана для одинаковых газов, но нормальная 
скорость газа снаружи стенки берется противоположной нормальной скорости газа внутри.
\[
\left\{
\tilde\rho, \tilde{u}_n, \tilde p
\right\} = 
\mathcal{R}_{1D}\Big(
\left\{
\rho, u_n, p, \gamma
\right\}_\text{in}
,
\left\{
\rho, -u_n, p, \gamma
\right\}_\text{in}
\Big)
\]
При этом, очевидно, $\tilde{u}_n = 0$, а $\tilde{\rho}$ и $\tilde{p}$ как-то зависят от параметров газа внутри 
трубы. Все компоненты нормального потока $\bvec{n}\vec{\bvec{F}}$ обнуляются, кроме компоненты нормального импульса, 
где в нормальном потоке остается член $\tilde{p}$. Величина $\tilde{p}$ может быть грубо оценена (из размерных соображений)
\[
\tilde{p} = p \pm \rho u_n^2,
\]
где <<$+$>> соответствует $u_n > 0$, а <<$-$>> --- случаю $u_n < 0$.

Стало быть, поперечный импульс эволюционирует по закону
\[
\pd{\rho |\bvec{u}_{\tau}|}{t} = -C\frac{\rho |\bvec{u}_\tau|^2}{h},
\]
где $C = O(1),\; h$ --- характерный диаметр трубы.
При выводе этого уравнения игнорировался приток и отток поперечного импульса вдоль трубы. Характерное время 
релаксации поперечного импульса $t_\text{relax} = \frac{h}{|\bvec{u}_\tau|}$.

Идея перехода к одномерной области состоит в устремлении $t_\text{relax} \rightarrow 0$. То есть газ, попадая 
в одномерную трубу теряет свой поперечный импульс мгновенно. Этого же эффекта можно добиться вообще не добавляя 
поперечный импульс в одномерные ячейки, то есть обнулить соответствующие компоненты нормального потока вдоль трубы.
Поскольку поперечная скорость газа в трубе теперь всегда будет равна нулю, граничные условия на стенках трубы 
более не требуются, т.к. поток отражения теперь в точности равен нулю.

Практически, между одномерными и трехмерными областями считается тот же самый нормальный поток, что и между
трехмерными областями, но поток поперечного импульса просто не добавляется к поперечному импульсу одномерной ячейки
\begin{gather*}
\bvec{n}\bvec{F}^{3D} = 
\begin{pmatrix}
\rho_1 u_n & \dots & \rho_S u_n &
 \rho u_n^2 + p & 
 \rho u_n u_{\tau,1}\phantom{0} & 
 \phantom{0}\rho u_n u_{\tau,2} & 
(\rho E + p) u_n
\end{pmatrix}\\
\bvec{n}\bvec{F}^{1D} = 
\begin{pmatrix}
\rho_1 u_n & \dots & \rho_S u_n &
 \rho u_n^2 + p & 
 \phantom{\rho u_n}0\phantom{u_{\tau,2}} & 
 \phantom{\rho u_n}0\phantom{u_{\tau,2}} & 
 (\rho E + p) u_n
\end{pmatrix}
\end{gather*}

\section{Реконструкция параметров в ячейке}
Предположим, что в ячейке параметры удовлетворяют стационарным уравнениям газовой динамики
\begin{equation}
\begin{gathered}
\pd{\rho u_z}{z} = 0\\
\pd{}{z}\left[\rho u_z^2 + p\right] = -\rho g\\
\pd{}{z}\left[(\rho E + p) u_z\right] = -\rho g u_z.
\end{gathered}
\label{eq:reconst}
\end{equation}
В дополнение к ним предположим $u_\bot = \operatorname{const}$. Интегральные средние для
массы, импульса и полной энергии известны:
\begin{gather*}
\frac{1}{h}\int\limits_{-h/2}^{h/2} \rho dz = \overline\rho\\
\frac{1}{h}\int\limits_{-h/2}^{h/2} \rho u_z dz = \overline{\rho u_z}\\
\frac{1}{h}\int\limits_{-h/2}^{h/2} \rho E dz = \overline{\rho E}.
\end{gather*}
Предполагается, что ячейка имеет высоту $h$, а ее центр имеет $z$-координату равную $0$.
Введем переменную $c^2 = \frac{\gamma p}{\rho}$ имеющую смысл квадрата скорости звука. При этом 
система уравнений \eqref{eq:reconst} принимает вид 
\begin{equation}
\begin{gathered}
\pd{\rho u_z}{z} = 0\\
\pd{}{z}\left[\rho \left(u_z^2 + \frac{c^2}{\gamma}\right)\right] = -\rho g\\
\pd{}{z}\left[\rho \left(\frac{u_\bot^2 + u_z^2}{2} + \frac{c^2}{\gamma-1}\right) u_z\right] = 
-\rho g u_z.
\end{gathered}
\label{eq:reconst2}
\end{equation}
Поскольку $\rho u_z$ и $u_\bot^2$ не зависят от $z$, последнее уравнение упрощается до
\[
\pd{}{z}\left[\frac{u_z^2}{2} + \frac{c^2}{\gamma-1}\right] = 
-g.
\]
Выражая $c^2 = (\gamma - 1) \left(A - g z - \frac{u_z^2}{2}\right)$
и подставляя $u_z = \frac{\overline{\rho u_z}}{\rho}$, получаем одно уравнение для $\rho$:
\begin{gather*}
\pd{}{z}\left[
\frac{\gamma + 1}{2\gamma}\frac{(\overline{\rho u_z})^2}{\rho(z)}
+\frac{A - gz}{\gamma}\rho(z)
\right] = -\rho(z) g
\end{gather*}
Решение данного уравнения приводит к трансцендентному уравнению, в которое входят два параметра, 
которые определяются из значений $\overline{\rho}, \overline{\rho u_z}, \overline{\rho E}$. Данный 
путь непригоден для реализации.

Вернемся к системе уравнений \eqref{eq:reconst2}:
\begin{equation}
\begin{gathered}
\pd{\rho u_z}{z} = 0\\
\pd{}{z}\left[\rho \left(u_z^2 + \frac{c^2}{\gamma}\right)\right] = -\rho g\\
\pd{}{z}\left[\frac{u_z^2}{2} + \frac{c^2}{\gamma-1}\right] = -g.
\end{gathered}
\label{eq:reconst3}
\end{equation}
Перейдем к производным $\pd{\rho}{z}, \pd{u_z}{z}, \pd{c}{z}$:
\begin{equation}
\left\{
\begin{array}{@{}c@{}c@{}c@{}c@{}c@{}c@{}c}
u_z\pd{\rho}{z} &{}+{}&\rho \pd{u_z}{z} &&&{}={}& 0\\
\left(u_z^2 + \frac{c^2}{\gamma}\right)\pd{\rho}{z} &{}+{}&
2\rho u_z \pd{u_z}{z} &{}+{}&
\frac{2\rho c}{\gamma} \pd{c}{z} &{}={}& -\rho g\\
&&u_z \pd{u_z}{z} &{}+{}&\frac{2c}{\gamma- 1} \pd{c}{z}&{}={}& -g.
\end{array}
\right.
\label{eq:reconst4}
\end{equation}
Упростим систему:
\begin{equation}
\left\{
\begin{array}{@{}c@{}c@{}c@{}c@{}c@{}c@{}c}
(\gamma - 1)u_z^2\pd{\rho}{z} &{}+{}&(\gamma - 1)\rho u_z\pd{u_z}{z} &&&{}={}& 0\\
\left(c^2-\gamma u_z^2\right)\pd{\rho}{z} &{}+{}&
&&
{2\rho c}\pd{c}{z} &{}={}& -\gamma\rho g\\
&&(\gamma-1)\rho u_z \pd{u_z}{z} &{}+{}&{2\rho c} \pd{c}{z}&{}={}& (1 - \gamma)\rho 
g.
\end{array}
\right.
\label{eq:reconst5}
\end{equation}
Решение данной системы относительно производных имеет вид
\begin{equation}
\begin{gathered}
\pd{\ln \rho}{z} = -\frac{g}{c^2 - u_z^2}\\
\pd{\ln u_z}{z} = \frac{g}{c^2 - u_z^2}\\
\pd{\ln c}{z} = -\frac{\gamma-1}{2}\frac{g}{c^2 - u_z^2}
\label{eq:sol}
\end{gathered}
\end{equation}
Примем предположение о постоянности величины $k = \frac{g}{c^2 - u_z^2} = \operatorname{const}$. 
Тогда
\begin{equation}
\begin{aligned}
\rho &{}= \rho_0 e^{-kz}\\
u_z &{}= u_{z,0} e^{kz}\\
\varepsilon &{}= \varepsilon_0 e^{(1-\gamma)kz}.
\label{eq:sol2}
\end{aligned}
\end{equation}
Среднее значение величины $A = A_0 e^{\beta z}$ равно
\[
\begin{gathered}
\overline{A} \equiv \frac{1}{h} \int\limits_{-h/2}^{h/2} A_0 e^{\beta z} dz = \frac{2A_0}{\beta h} 
\frac{e^{\beta h/2} - e^{-\beta h/2}}{2} = A_0 \frac{\sh \frac{\beta h}{2}}{\frac{\beta h}{2}} 
\equiv A_0 \operatorname{shc} 
\frac{\beta h}{2}\\
A_0 = \frac{\overline{A}}{\operatorname{shc} \frac{\beta h}{2}}
\end{gathered}
\]

Алгоритм реконструкции:
\begin{enumerate}
\item Вычисляем $k$ по средним значениям в ячейке
\[
k = \frac{2\overline\rho^2 g}{2\gamma(\gamma-1)\overline\rho\overline{\rho E} - (\gamma^2 - \gamma 
+ 2) (\overline{\rho u_z})^2}.
\]
\item Находим $\rho_0, u_{z,0}$ по формулам
\[
\rho_0 = \frac{\overline{\rho}}{\operatorname{shc} \frac{kh}{2}}, \qquad u_{z,0} = 
\frac{\overline{\rho 
u_z}}{\rho_0}.
\]
\item Найдем $\varepsilon_0$. Выразим среднее значение $\rho E$
\begin{multline*}
\overline{\rho E} \equiv
\frac{1}{h}\int\limits_{-h/2}^{h/2} \rho E dz = 
\frac{1}{h}\int\limits_{-h/2}^{h/2} \rho \left(\frac{u_z^2}{2} + \frac{u_\bot^2}{2} + 
\varepsilon\right) dz = \\ = \overline{\rho} \frac{u_{z,0}^2 + u_\bot^2}{2}
+\rho_0\varepsilon_0 \operatorname{shc} \frac{\gamma k h}{2}.
\end{multline*}
Отсюда легко получить выражение для $\varepsilon_0$:
\[
\varepsilon_0 = \frac{2\overline{\rho E} - \overline{\rho} (u_{z,0}^2 + u_{\bot}^2)}{2\rho_0 
\operatorname{shc} \frac{\gamma kh}{2}}
\]
\end{enumerate}

\end{document}
