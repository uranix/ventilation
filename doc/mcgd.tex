\documentclass[12pt]{article}
\usepackage[utf8]{inputenc}
\usepackage[T2A]{fontenc}
\usepackage[english,russian]{babel}
\usepackage{amsmath,amssymb}
\usepackage{indentfirst}

\newcommand{\pd}[2]{\frac{\partial #1}{\partial #2}}
\newcommand{\dpd}[2]{\dfrac{\partial #1}{\partial #2}}
\renewcommand{\arraystretch}{1.2}
\let\dividesymbol\div
\renewcommand{\div}{\operatorname{div}}
\newcommand{\grad}{\operatorname{grad}}
\newcommand{\bvec}[1]{\boldsymbol{\mathbf{#1}}}
\newcommand{\cutefrac}[2]{{}^{#1}\mkern-5mu/{\!}_#2}
\newcommand{\half}{{\cutefrac{1}{2}}}
\newcommand{\tr}{\mathsf{T}}
\renewcommand{\iiint}{\int\hspace{-10pt}\int\hspace{-10pt}\int}
\newcommand{\oiint}{%
{}\subset\hspace{-6pt}\supset\hspace{-17.8pt}%
\int\hspace{-10pt}\int%
}%

\author{Цыбулин Иван}
\title{Система уравнений многокомпонентной газовой динамики}

\begin{document}
\maketitle
\section{Обозначения}
\begin{itemize}
\item $\rho_i$ --- плотность $i$-й компоненты.
\item $M_i$ --- молярная масса компоненты $i$.
\item $\rho \equiv \sum \rho_i$ --- плотность многокомпонентной смеси.
\item $\theta_i \equiv \frac{\rho_i}{\rho}$ --- массовая доля компоненты $i$ в смеси.
\item $x_i$ --- молярная доля компоненты $i$ в смеси.
\item $\bvec q = (u, v, w)$ --- скорость газа.
\item $\gamma_i$ --- показатель адиабаты компоненты $i$.
\item $\beta_i \equiv \frac{1}{\gamma_i - 1}$ --- молярная теплоемкость при постоянном объеме 
компоненты $i$, выраженная в единицах $R$.
\item $T$ --- температура.
\item $p$ --- давление.
\item $\varepsilon$ --- внутренняя энергия газовой смеси.
\item $e = \rho \left(\varepsilon + \frac{q^2}{2}\right)$ --- полная энергия смеси.
\end{itemize}

\subsection{Усредненные величины}
В элементарном объеме $V$ находится масса $\rho_i V$ $i$-й компоненты или $\frac{\rho_i V}{M_i}$ 
моль этой компоненты. Средняя молярная масса газа
\[
M = \frac{\sum_i \rho_i V}{\sum_i \rho_i V / M_i} = \frac{\rho}{\sum_i \rho_i / M_i}.
\]
\[
\frac{\rho}{M} = \sum_i \frac{\rho_i}{M_i} = \rho\sum_i\frac{\theta_i}{M_i}.
\]
\[
\frac{1}{M} = \sum_i \frac{\theta_i}{M_i}
\]
Теплоемкость при постоянном объеме одного моля смеси
\[
\beta R = c_v = \sum_i x_i c_{v,i}
= \sum_i x_i \beta_i R
\]
\[
\beta = \frac{\sum_i \beta_i \theta_i / M_i}{\sum_i \theta_i / M_i} = 
M \sum_i \beta_i \frac{\theta_i}{M_i}
\]
\[
\frac{\beta}{M} = \sum_i \frac{\beta_i}{M_i} \theta_i
\]
Легко видеть, что показатель адиабаты $\gamma$ смеси
\[
\frac{\gamma \beta}{M} = \sum_i \frac{\beta_i \gamma_i}{M_i} \theta_i = 
\sum_i \frac{1 + \beta_i}{M_i} \theta_i = \frac{1}{M} + \frac{\beta}{M} = \frac{1+\beta}{M}
\]
связан с $\beta$ тем же соотношением
\[
\beta = \frac{1}{\gamma - 1}.
\]
И $\beta$ и $\gamma$ являются функциями исключительно $\theta_i$.

\section{Система уравнений}
Система уравнений в консервативной форме ($n$ - число компонент)
\begin{gather*}
\pd{\bvec U(\bvec V)}{t} + \pd{\bvec F(\bvec V)}{x} = 0\\
\bvec V = \begin{pmatrix}
\rho & \theta_2 & \theta_3 & \dots & \theta_n & u & v & w & c
\end{pmatrix}\\
\bvec U = \begin{pmatrix}
\rho & \rho_2 & \rho_3 & \dots & \rho_n & \rho u & \rho v & \rho w & e
\end{pmatrix}\\
\bvec F = \begin{pmatrix}
\rho u & \rho_2 u & \rho_3 u & \dots & \rho_n u & \rho u^2 + p & \rho u v & \rho u w & (e + p)u
\end{pmatrix}
\end{gather*}

Выражение консервативных величин через неконсервативные
\begin{gather*}
\rho_i = \rho \theta_i\\
q^2 = u^2 + v^2 + w^2\\
e = \rho \left(\frac{\beta c^2}{\gamma}+\frac{q^2}{2}\right)\\
p = \frac{\rho c^2}{\gamma}\\
e + p = \rho \left(\beta c^2+\frac{q^2}{2}\right).
\end{gather*}

\subsection{Спектральное разложение}

Матрица $\bvec A = \pd{\bvec F}{\bvec V} \left(\pd{\bvec U}{\bvec V}\right)^{-1}$. 
\begin{gather*}
\bvec \Omega \bvec A = \bvec \Lambda \bvec \Omega\\
\bvec \Omega \pd{\bvec F}{\bvec V} 
= \bvec \Lambda \bvec \Omega\pd{\bvec U}{\bvec V} \\
\bvec \omega_i^\tr \pd{\bvec F}{\bvec V} 
= \lambda_i \bvec \omega_i \pd{\bvec U}{\bvec V} \\
\bvec \omega_i^\tr \left(\pd{\bvec F}{\bvec V} - \lambda_i \pd{\bvec U}{\bvec V}\right) = \bvec 0\\
\end{gather*}
\[
\pd{\bvec U}{\bvec V} = 
\begin{pmatrix}
1 & 0 & 0 & \dots & 0 & 0 & 0 & 0 & 0\\
\theta_2 & \rho & 0 & \dots & 0 & 0 & 0 & 0 & 0\\
\theta_3 & 0 & \rho & \dots & 0 & 0 & 0 & 0 & 0\\
 &  &  & \ddots\\
\theta_n & 0 & 0 & \dots & \rho & 0 & 0 & 0 & 0\\ 
u & 0 & 0 & \dots & 0 & \rho & 0 & 0 & 0\\ 
v & 0 & 0 & \dots & 0 & 0 & \rho & 0 & 0\\ 
w & 0 & 0 & \dots & 0 & 0 & 0 & \rho & 0\\ 
e / \rho & \rho c^2 \pd{(\beta/\gamma)}{\theta_2} & \rho c^2 \pd{(\beta/\gamma)}{\theta_3} & \dots 
& \rho c^2 \pd{(\beta/\gamma)}{\theta_n} & \rho u & \rho v & \rho w & 2\rho c \beta/\gamma
\end{pmatrix}
\]
\[
\pd{\bvec F}{\bvec V} = 
\begin{pmatrix}
u & 0 & 0 & \dots & 0 & \rho & 0 & 0 & 0\\
u\theta_2 & \rho u & 0 & \dots & 0 & \rho_2 & 0 & 0 & 0\\
u\theta_3 & 0 & \rho u & \dots & 0 & \rho_3 & 0 & 0 & 0\\
 &  &  & \ddots\\
u\theta_n & 0 & 0 & \dots & \rho u & \rho_n & 0 & 0 & 0\\ 
u^2 + \frac{c^2}{\gamma} & 
\rho c^2\pd{(1/\gamma)}{\theta_2} & 
\rho c^2\pd{(1/\gamma)}{\theta_3} & 
\dots & 
\rho c^2\pd{(1/\gamma)}{\theta_n} & 
2\rho u & 0 & 0 & \frac{2\rho c}{\gamma} \\ 
u v & 0 & 0 & \dots & 0 & \rho v & \rho u & 0 & 0\\ 
u w & 0 & 0 & \dots & 0 & \rho w & 0 & \rho u & 0\\ 
\frac{p+e}{\rho}u & 
\rho u c^2 \pd{\beta}{\theta_2} & 
\rho u c^2 \pd{\beta}{\theta_3} & \dots & 
\rho u c^2 \pd{\beta}{\theta_n} & p + e + \rho u^2 & \rho uv & \rho uw & 2\rho c u\beta
\end{pmatrix}
\]
Пусть $\lambda_i = u + s_i$. Тогда
\[
\operatorname{det}\left(
\pd{\bvec F}{\bvec V} - \lambda_i \pd{\bvec U}{\bvec V}
\right) =
\operatorname{det}\left(
\pd{\bvec F}{\bvec V} - u \pd{\bvec U}{\bvec V} - s_i \pd{\bvec U}{\bvec V}
\right) = 
\operatorname{det}\left(
\bvec G - s_i \pd{\bvec U}{\bvec V}
\right)
\]
\[
\bvec G = \pd{\bvec F}{\bvec V} - u \pd{\bvec U}{\bvec V} =
\begin{pmatrix}
0 & 0 & 0 & \dots & 0 & \rho & 0 & 0 & 0\\
0 & 0 & 0 & \dots & 0 & \rho_2 & 0 & 0 & 0\\
0 & 0 & 0 & \dots & 0 & \rho_3 & 0 & 0 & 0\\
 &  &  & \ddots\\
0 & 0 & 0 & \dots & 0 & \rho_n & 0 & 0 & 0\\ 
\frac{c^2}{\gamma} & 
\rho c^2\pd{(1/\gamma)}{\theta_2} & 
\rho c^2\pd{(1/\gamma)}{\theta_3} & 
\dots & 
\rho c^2\pd{(1/\gamma)}{\theta_n} & 
\rho u & 0 & 0 & \frac{2\rho c}{\gamma} \\ 
0 & 0 & 0 & \dots & 0 & \rho v & 0 & 0 & 0\\ 
0 & 0 & 0 & \dots & 0 & \rho w & 0 & 0 & 0\\ 
\frac{uc^2}{\gamma} & 
\rho u c^2 \pd{(1/\gamma)}{\theta_2} & 
\rho u c^2 \pd{(1/\gamma)}{\theta_3} & \dots & 
\rho u c^2 \pd{(1/\gamma)}{\theta_n} & p + e & 0 & 0 & \frac{2\rho c u}{\gamma}
\end{pmatrix}
\]
Матрица $\bvec G$ имеет размер $n+4 \times n+4$. У этой матрицы $n+2$ левых собственных вектора, 
соответствующих $s_i = 0$.
\[
\bvec \Omega_{0} \bvec G = \bvec 0.
\]
\[
\bvec \Omega_{0} = \begin{pmatrix}
\theta_2 & -1 & 0 & \dots & 0 & 0 & 0 & 0 & 0\\
\theta_3 & 0 & -1 & \dots & 0 & 0 & 0 & 0 & 0\\
&&&\ddots\\
\theta_n & 0 & 0 & \dots & -1 & 0 & 0 & 0 & 0\\
u^2 - \beta c^2 - \frac{q^2}{2}  &
0 & 0 & \dots & 0 & -u & 0 & 0 & 1\\
v & 0 & 0 & \dots & 0 & 0 & -1 & 0 & 0\\
w & 0 & 0 & \dots & 0 & 0 & 0 & -1 & 0
\end{pmatrix}
\]
Остальные два собственных вектора матрицы $\pd{\bvec F}{\bvec U}$
\begin{gather*}
s_{+} = +c\\
\bvec \omega^\tr_{+} = 
\begin{pmatrix}
- \frac{q^2}{2} + \beta c u - \frac{c^2}{\gamma} 
\sum_{i=2}^n \theta_i \pd{\beta}{\theta_i}
,&
\frac{c^2}{\gamma}\pd{\beta}{\theta_2},&
\frac{c^2}{\gamma}\pd{\beta}{\theta_3},&
\dots,&
\frac{c^2}{\gamma}\pd{\beta}{\theta_n},&
u - \beta c, & v, & w, & -1
\end{pmatrix}\\
s_{-} = -c\\
\bvec \omega^\tr_{-} = 
\begin{pmatrix}
- \frac{q^2}{2} - \beta c u - \frac{c^2}{\gamma} 
\sum_{i=2}^n \theta_i \pd{\beta}{\theta_i}
,&
\frac{c^2}{\gamma}\pd{\beta}{\theta_2},&
\frac{c^2}{\gamma}\pd{\beta}{\theta_3},&
\dots,&
\frac{c^2}{\gamma}\pd{\beta}{\theta_n},&
u + \beta c, & v, & w, & -1
\end{pmatrix}
\end{gather*}

Матрица левых собственных векторов
\[
\bvec \Omega =
\begin{pmatrix}
\theta_2 & -1 & 0 & \dots & 0 & 0 & 0 & 0 & 0\\
\theta_3 & 0 & -1 & \dots & 0 & 0 & 0 & 0 & 0\\
&&&\ddots\\
\theta_n & 0 & 0 & \dots & -1 & 0 & 0 & 0 & 0\\
u^2 - \beta c^2 - \frac{q^2}{2} &
0 & 0 & \dots & 0 & -u & 0 & 0 & 1\\
v & 0 & 0 & \dots & 0 & 0 & -1 & 0 & 0\\
w & 0 & 0 & \dots & 0 & 0 & 0 & -1 & 0\\
- \frac{q^2}{2} + \beta c u - \frac{c^2}{\gamma} 
\sum_{i=2}^n \theta_i \pd{\beta}{\theta_i}
&
\frac{c^2}{\gamma}\pd{\beta}{\theta_2}&
\frac{c^2}{\gamma}\pd{\beta}{\theta_3}&
\dots&
\frac{c^2}{\gamma}\pd{\beta}{\theta_n}&
u - \beta c & v & w & -1\\
- \frac{q^2}{2} - \beta c u - \frac{c^2}{\gamma} 
\sum_{i=2}^n \theta_i \pd{\beta}{\theta_i}
&
\frac{c^2}{\gamma}\pd{\beta}{\theta_2}&
\frac{c^2}{\gamma}\pd{\beta}{\theta_3}&
\dots&
\frac{c^2}{\gamma}\pd{\beta}{\theta_n}&
u + \beta c & v & w & -1
\end{pmatrix}
\]
Определитель матрицы $\bvec \Omega$ равен $2\beta^2 c^3$.
Обратную матрицу можно представить компактно, если ввести вектора
\[
\bvec r = \begin{pmatrix}
1 & \theta_2 & \theta_3 & \dots & \theta_n &
u & v & w & \frac{q^2}{2} + \beta c^2
\end{pmatrix}^\tr
\]
и
\[
\bvec \ell = \begin{pmatrix}
\frac{c^2}{\gamma}\pd{\beta}{\theta_2}&
\frac{c^2}{\gamma}\pd{\beta}{\theta_3}&
\dots&
\frac{c^2}{\gamma}\pd{\beta}{\theta_n}&
1 & v & w & \frac{1}{2} & \frac{1}{2}
\end{pmatrix}.
\]
Обратная матрица представима в виде $\bvec \Omega^{-1} = \bvec T - \frac{1}{\beta c^2} \bvec r 
\bvec \ell$, где
\[
\bvec T = \begin{pmatrix}
 0 & 0 & \dots & 0 & 0 & 0 & 0 & 0 & 0\\
-1 & 0 & \dots & 0 & 0 & 0 & 0 & 0 & 0\\
0 & -1 & \dots & 0 & 0 & 0 & 0 & 0 & 0\\
&& \ddots\\
0&0&\dots&-1& 0 & 0 & 0 & 0 & 0\\
0&0&\dots& 0& 0 & 0 & 0 & -\frac{1}{2\beta c} & \frac{1}{2\beta c}\\
0&0&\dots& 0& 0 & -1 & 0 & 0 & 0\\
0&0&\dots& 0& 0 & 0 & -1 & 0 & 0\\
0&0&\dots& 0& 1 & 0 & 0 & -\frac{u}{2\beta c} & \frac{u}{2\beta c}
\end{pmatrix}.
\]
Собственные значения матрицы $\bvec A$, соответствующие строкам матрицы $\bvec \Omega$ равны
\[
\bvec \Lambda = \operatorname{diag}(u, u, \dots, u, u + c, u - c).
\]
\section{Производные}
Учтем, что $\theta_1 = 1 - \sum_{i = 2}^n \theta_i$ и 
\begin{gather*}
\frac{\beta}{M} = \sum_i \frac{\beta_i\theta_i}{M_i} = 
\frac{1 - \sum_{i=2}^n \theta_i}{M_1}\beta_1
+\sum_{i=2}^n \frac{\beta_i\theta_i}{M_i}\\
\frac{1}{M} = \sum_i \frac{\theta_i}{M_i} = 
\frac{1 - \sum_{i=2}^n \theta_i}{M_1}
+\sum_{i=2}^n \frac{\theta_i}{M_i}
\end{gather*}
\begin{multline*}
\pd{\beta}{\theta_i} = 
\pd{}{\theta_i}\left(\frac{\beta}{M}\right)M
+\frac{\beta M}{M^2} \pd{M}{\theta_i} =
\pd{}{\theta_i}\left(\frac{\beta}{M}\right)M
-\beta M \pd{}{\theta_i}\left(\frac{1}{M}\right) = \\
\left(\frac{\beta_i}{M_i}-\frac{\beta_1}{M_1}\right)M + 
M \left(\frac{\beta}{M_1} - \frac{\beta}{M_i}\right) = 
M \left(\frac{\beta - \beta_1}{M_1} + \frac{\beta_i - \beta}{M_i}\right).
\end{multline*}

\end{document}
